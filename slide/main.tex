\documentclass[aspectratio=169]{beamer}  % 16:9 aspect ratio

% Use a clean theme as base
\usetheme{default}
\usecolortheme{default}
\usepackage{booktabs}

% Custom colors from HKUST logo
\definecolor{hkustblue}{RGB}{0, 51, 119}    % Navy blue from logo
\definecolor{hkustgold}{RGB}{180, 141, 61}  % Golden brown from logo
\definecolor{lightgray}{RGB}{236, 240, 241}

% Customize the appearance
\setbeamercolor{structure}{fg=hkustblue}
\setbeamercolor{background canvas}{bg=white}
\setbeamercolor{normal text}{fg=hkustblue}
\setbeamercolor{frametitle}{fg=hkustblue,bg=white}
\setbeamercolor{itemize item}{fg=hkustgold}
\setbeamercolor{itemize subitem}{fg=hkustgold}
\setbeamercolor{block title}{fg=white,bg=hkustblue}
\setbeamercolor{block body}{fg=hkustblue,bg=lightgray}
\setbeamercolor{title}{fg=hkustblue}
\setbeamercolor{subtitle}{fg=hkustgold}

% Remove navigation symbols
\setbeamertemplate{navigation symbols}{}

% Customize frame title
\setbeamertemplate{frametitle}{
    \vspace*{0.5cm}
    \insertframetitle
    \vspace*{0.2cm}
    \begin{beamercolorbox}[wd=\paperwidth,ht=0.2pt]{structure}
    \end{beamercolorbox}
}

% Customize itemize bullets
\setbeamertemplate{itemize item}{\small\raise0.5pt\hbox{\textbullet}}
\setbeamertemplate{itemize subitem}{\tiny\raise1.5pt\hbox{\textbullet}}

% Packages
\usepackage{graphicx}
\usepackage{amsmath}
\usepackage{hyperref}
% add page number
\setbeamertemplate{footline}[frame number]

% Title page information
\title{Structural Analysis of Vertical Contracting}
\subtitle{Empirical Framework}
\author{Siyuan (Bruce) Jin,\\ \small{3rd Year PhD Candidate in Information Systems}\\ \small{\href{https://siyuan-bruce.github.io}{siyuan-bruce.github.io}}}
\institute{Hong Kong University of Science and Technology}
\date{March, 2025}

\begin{document}

% Title page
\begin{frame}
    \titlepage
\end{frame}

% Slide 2: Two-Stage Framework
\begin{frame}{Two-Stage Framework for Analysis}
    \begin{itemize}
        \item Researchers often model vertical contracting using a two-stage framework.
        \item Stage 1: Supply
        \begin{itemize}
            \item Firms negotiate contracts and take payoff-relevant actions (e.g., investment, pricing).
        \end{itemize}
        \item Stage 2: Demand
        \begin{itemize}
            \item Consumers purchase products/services provided by upstream and downstream firms.
        \end{itemize}
    \end{itemize}
\end{frame}

% Slide 5: Contracts
\begin{frame}{Contracts}
    \begin{itemize}
        \item Contracts represent agreements between upstream and downstream firms.
        \item Denoted as:
        \[
        C_{ij} \in \mathbb{C}
        \]
        \item Where:
        \begin{itemize}
            \item \(C_{ij}\): Contract between upstream firm \(i\) and downstream firm \(j\).
            \item \(\mathbb{C}\): Set of feasible contracts.
        \end{itemize}
        \item Null contract (\(C_0\)) represents the disagreement (no contract) outcome.
    \end{itemize}
\end{frame}

% Slide 6: Payoff-Relevant Actions
\begin{frame}{Payoff-Relevant Actions}
    \begin{itemize}
        \item Actions not explicitly specified in contracts but affect payoffs.
        \item Denoted by:
        \[
        a = \{a_0, a_1(\mathbb{C}, a_0)\}
        \]
        \item Where:
        \begin{itemize}
            \item \(a_0\): Actions chosen simultaneously with contracts.
            \item \(a_1(\mathbb{C}, a_0)\): Actions chosen after contracts and initial actions.
        \end{itemize}
        \item Examples:
        \begin{itemize}
            \item Downstream pricing, product availability.
            \item Effort provision or investment.
        \end{itemize}
    \end{itemize}
\end{frame}

% Slide 7: Payoff Representation
\begin{frame}{Payoff Representation}
    \textbf{Detailed Representation:}
    \begin{itemize}
        \item Payoffs for each upstream firm \(i\) and downstream firm \(j\) are represented as:
        \[
        \Pi_{U_i}(\mathbb{C}, \boldsymbol{a}_0), \quad \Pi_{D_j}(\mathbb{C}, \boldsymbol{a}_0)
        \]
        \item These payoffs implicitly depend on:
        \begin{itemize}
            \item Subsequent actions taken by firms (\(\boldsymbol{a}_1(\cdot)\)).
            \item Consumer actions captured by demand functions.
        \end{itemize}
    \end{itemize}

    \vspace{0.3cm}
    \textbf{Demand Representation:}
    \begin{itemize}
        \item Upstream demand: \(\bar{D}(\cdot)\).
        \item Downstream demand: \(\underline{D}(\cdot)\).
    \end{itemize}
\end{frame}


\begin{frame}{Payoff Representation}
  
    \textbf{Demand Representation:}
    \begin{itemize}
        \item Example (successive monopoly with per-unit pricing):
        \[
        \mathbb{C} = \{w\}, \quad a_0 = \emptyset, \quad a_1 = \{p\}
        \]
        \[
        \bar{D}(w) = \underline{D}(w) = D\left(p^m\left(w + c_R\right)\right)
        \]
        \item \{w\}: upstream price;
        \item \{p\}: downstream price;
        \item $p^m$: monopoly price coefficient;
        \item $c_R$: retailer's marginal cost.
        \item In this case, upstream and downstream demand coincide because of a single retailer-manufacturer setup.
    \end{itemize}

    \vspace{0.3cm}
    \textbf{General Case:}
    \begin{itemize}
        \item When there are multiple upstream firms or consumers do not always purchase upstream products, \(\bar{D}(\cdot)\) and \(\underline{D}(\cdot)\) will typically differ.
    \end{itemize}
\end{frame}

% Slide 8: Example - Medical Devices
\begin{frame}{Example: Medical Devices (Grennan, 2013)}
    \begin{itemize}
        \item Contracts \(\mathbb{C}\) between stent manufacturers (\(i\)) and hospitals (\(j\)) specify linear prices \(\boldsymbol{w} = \{w_{ij}\}\).
        \item The null contract \(\mathbb{C}_0\) is represented by \(w_{ij} = \infty\), meaning no trade occurs.
    \end{itemize}
\end{frame}

\begin{frame}{Example: Medical Devices (Grennan, 2013)}
    \begin{itemize}
        
        \item The payoff for stent manufacturer \(i\) is:
        \[
        \Pi_{U_i}(\mathbb{C}) = \sum_{j : \mathbb{C}_{ij} \neq \mathbb{C}_0} \left(w_{ij} - c_i\right) \bar{D}_{ij}\left(\{w_{kj}\}_{k \in \mathcal{I}}\right)
        \]
        \begin{itemize}
            \item \(w_{ij}\) is the price in the contract between manufacturer \(i\) and hospital \(j\).
            \item \(c_i\) is the marginal cost of producing stent \(i\).
            \item \(\bar{D}_{ij}(\cdot)\) is the quantity of stent \(i\) used at hospital \(j\), which depends on:
            \begin{itemize}
                \item Preferences of doctors and patients at hospital \(j\).
                \item Contracts \(\{w_{kj}\}\) for all stents signed by hospital \(j\).
            \end{itemize}
        \end{itemize}

        \item The payoff for hospital \(j\) is:
        \[
        \Pi_{D_j}(\mathbb{C}) = W_j\left(\{w_{kj}\}_{k \in \mathcal{I}}\right)
        \]
        \begin{itemize}
            \item \(W_j(\cdot)\) is the welfare of hospital \(j\), which depends on the prices \(\{w_{kj}\}\) for all stents it uses.
        \end{itemize}
    \end{itemize}
\end{frame}

% Slide 1: Commercial Health Insurers and Hospitals - Setup
\begin{frame}{Example:  Health Insurers and Hospitals (Ho and Lee, 2017)}
    \begin{itemize}
        \item Contracts between hospitals (upstream) and insurers (downstream) in the U.S. healthcare industry.
        \item Contracts \(\mathbb{C}\) specify payments per hospital admission:
        \[
        \boldsymbol{w} = \{w_{ij}\}
        \]
        \item Insurers also set premiums for households:
        \[
        \boldsymbol{p} = \{p_j\}
        \]
        \item Demand terms:
        \begin{itemize}
            \item \(\underline{D}_j(\boldsymbol{p}, \boldsymbol{N})\): Households enrolled in insurer \(j\).
            \item \(\bar{D}_{ij}(\boldsymbol{p}, \boldsymbol{N})\): Admissions from insurer \(j\)'s enrollees to hospital \(i\).
        \end{itemize}
        \item \(\boldsymbol{N} = \{ij : \mathbb{C}_{ij} \neq \mathbb{C}_0\}\) represents the network of contracts.
    \end{itemize}
\end{frame}

% Slide 2: Commercial Health Insurers and Hospitals - Payoffs
\begin{frame}{Example: Health Insurers and Hospitals (Ho and Lee, 2017)}
    \begin{itemize}
        \item Hospital \(i\)'s payoff:
        \[
        \Pi_{U_i}(\mathbb{C}, \boldsymbol{p}) = \sum_{j : \mathbb{C}_{ij} \neq \mathbb{C}_0} \left(w_{ij} - c_i\right) \bar{D}_{ij}(\boldsymbol{p}, \boldsymbol{N})
        \]
        \begin{itemize}
            \item \(w_{ij}\): Payment per admission.
            \item \(c_i\): Per-admission cost.
        \end{itemize}
        
        \item Insurer \(j\)'s payoff:
        \[
        \Pi_{D_j}(\mathbb{C}, \boldsymbol{p}) = \left(p_j - \eta_j\right) \underline{D}_j(\boldsymbol{p}, \boldsymbol{N}) - \sum_{i : \mathbb{C}_{ij} \neq \mathbb{C}_0} w_{ij} \bar{D}_{ij}(\boldsymbol{p}, \boldsymbol{N})
        \]
        \begin{itemize}
            \item \(p_j\): Insurer premium.
            \item \(\eta_j\): Non-hospital costs (e.g., physician or drug payments).
        \end{itemize}
        
        \item Differences from the previous example:
        \begin{itemize}
            \item Payoffs depend on premiums, an additional supply-side decision.
            \item Insurers compete for households, so all firms' actions affect payoffs.
        \end{itemize}
    \end{itemize}
\end{frame}


% Slide 1: Multichannel Television - Setup
\begin{frame}{Example: Multichannel Television and Vertical Integration}
    \begin{itemize}
        \item Crawford and Yurukoglu (2012) and Crawford et al. (2018) study negotiations between:
        \begin{itemize}
            \item Upstream television channels (\(i\)).
            \item Downstream multichannel video programming distributors (MVPDs, \(j\)), such as cable and satellite firms.
        \end{itemize}
        \item Contracts \(\mathbb{C}\) specify linear affiliate fees:
        \[
        \boldsymbol{w} = \{w_{ij}\}
        \]
        \item \(w_{ij}\): Amount distributor \(j\) pays channel \(i\) per subscriber.
        \item Distributors choose subscription prices:
        \[
        \boldsymbol{p} = \{p_j\}
        \]
    \end{itemize}
\end{frame}

% Slide 2: Multichannel Television - Timing and Demand
\begin{frame}{Timing and Demand in Multichannel Television}
    \begin{itemize}
        \item Demand objects:
        \begin{itemize}
            \item \(\underline{D}_j(\boldsymbol{p}, \boldsymbol{N})\): Number of households subscribing to distributor \(j\).
            \item \(\boldsymbol{N} = \{ij : \mathbb{C}_{ij} \neq \mathbb{C}_0\}\): Network of channel-distributor agreements.
        \end{itemize}
    \end{itemize}
\end{frame}

% Slide 3: Multichannel Television - Payoffs
\begin{frame}{Payoffs in Multichannel Television}
    \begin{itemize}
        \item Channel \(i\)'s payoff:
        \[
        \Pi_{U_i}(\mathbb{C}, \boldsymbol{p}) = \sum_{j : \mathbb{C}_{ij} \neq \mathbb{C}_0} \left(w_{ij} \underline{D}_j(\boldsymbol{p}, \boldsymbol{N}) + ad_{ij}(\boldsymbol{p}, \boldsymbol{N})\right)
        \]
        \begin{itemize}
            \item \(ad_{ij}(\cdot)\): Advertising revenue from distributor \(j\)'s subscribers.
            \item $w_{ij}$: channel fee received from distributor $j$.
        \end{itemize}

        \item Distributor \(j\)'s payoff:
        \[
        \Pi_{D_j}(\mathbb{C}, \boldsymbol{p}) = \left(p_j - \sum_{i : \mathbb{C}_{ij} \neq \mathbb{C}_0} w_{ij}\right) \underline{D}_j(\boldsymbol{p}, \boldsymbol{N})
        \]
        \begin{itemize}
            \item \(p_j\): Subscription price set by distributor \(j\).
            \item \(\underline{D}_j(\cdot)\): Number of households subscribing to distributor \(j\).
        \end{itemize}

    \end{itemize}
\end{frame}

\begin{frame}{Discussion}
    \begin{itemize}
            \item Key difference from Example 11:
        \begin{itemize}
            \item Here, upstream fees \(w_{ij}\) are paid for all subscribers (\(\underline{D}_j\)).
            \item In Example 11, upstream fees were paid only for specific hospital admissions (\(\bar{D}_{ij}\)).
        \end{itemize}
    \end{itemize}
\end{frame}

% Slide 13: Sequential vs Simultaneous Timing
\begin{frame}{Sequential vs Simultaneous Timing}
    \begin{itemize}
        \item Timing assumptions play a critical role in contracting models:
        \begin{itemize}
            \item Simultaneous: Actions like pricing and contracting are decided together.
            \item Sequential: Contracting concludes before other actions are taken (e.g., pricing).
        \end{itemize}
        \item Example: Multichannel TV contracts often assume sequential timing for pricing decisions.
    \end{itemize}
\end{frame}



% Slide 14: Modeling Contract Formation
\begin{frame}{Modeling Contract Formation}
\begin{itemize}
    \item \textbf{Different Approaches to Modeling Contract Formation:}
    \begin{enumerate}
        \item \textbf{Take-It-Or-Leave-It (TIOLI) Offers}
        
        \item \textbf{Nash-in-Nash Bargaining}
    \end{enumerate}
    \end{itemize}
\end{frame}


% Slide 14: Modeling Contract Formation
\begin{frame}{Take-It-Or-Leave-It (TIOLI) Offers}
\begin{itemize}
    \item In this framework, one side of the negotiation (e.g., the upstream firm, such as the manufacturer) \textbf{unilaterally proposes a contract offer} to the other side (e.g., the downstream firm, such as the retailer).
    \item The receiving party can either:
    \begin{itemize}
        \item \textbf{Accept the offer}, in which case the contract terms are implemented as proposed.
        \item \textbf{Reject the offer}, in which case no agreement is reached, and both sides receive their disagreement payoffs (e.g., profits they would earn without a deal).
    \end{itemize}
    \item This approach assumes one party has the \textbf{power to dictate the terms of the contract}.
    \item Example: Villas-Boas (2007) uses a TIOLI framework to analyze manufacturer-retailer contracts, examining how manufacturers' offers affect the retailer's decision-making.
    \item Strengths:
    \begin{itemize}
        \item Simple and easy to implement in theoretical and empirical models.
        \item Useful when one party dominates the bargaining process.
    \end{itemize}
\end{itemize}
\end{frame}



% Slide 14: Modeling Contract Formation
\begin{frame}{Nash-in-Nash Bargaining}
    \begin{itemize}
        \item In this framework, firms \textbf{bargain simultaneously} over contract terms.
        \item Each firm uses its \textbf{outside options and leverage} to negotiate favorable terms.
        \item Assumes mutual flexibility and the ability to reach efficient agreements.
        \item Useful for modeling industries where both upstream and downstream firms have significant bargaining power.
        \item Strengths:
        \begin{itemize}
            \item Captures mutual influence in negotiation dynamics.
            \item More flexible and realistic in industries with balanced power dynamics.
        \end{itemize}
        \item Weaknesses:
        \begin{itemize}
            \item Computational complexity in empirical applications.
            \item Requires detailed data to estimate bargaining power and outside options.
        \end{itemize}
    \end{itemize}
    \end{frame}
   
% Slide 16: Nash-in-Nash Bargaining Model
\begin{frame}{Nash-in-Nash Bargaining Model}
    \begin{itemize}
        \item Nash-in-Nash bargaining captures simultaneous negotiations between pairs of firms.
        \item Necessary condition:
\begin{footnotesize}
\[
\widehat{\mathbb{C}}_{ij} \in \arg \max_{\mathbb{C}_{ij} \in \mathcal{C}_{ij}^+(\widehat{\mathbb{C}}_{-ij})}  
\Bigg[
\underbrace{
    \Big(\Pi_{Dj}(\{\mathbb{C}_{ij}, \widehat{\mathbb{C}}_{-ij}\}) 
    - \Pi_{Dj}(\{\mathbb{C}_0, \widehat{\mathbb{C}}_{-ij}\})\Big)^{b_{ij}}
}_{\Delta_{ij} \Pi_{Dj}(\{\mathbb{C}_{ij}, \widehat{\mathbb{C}}_{-ij}\})}
\cdot
\underbrace{
    \Big(\Pi_{Ui}(\{\mathbb{C}_{ij}, \widehat{\mathbb{C}}_{-ij}\}) 
    - \Pi_{Ui}(\{\mathbb{C}_0, \widehat{\mathbb{C}}_{-ij}\})\Big)^{1-b_{ij}}
}_{\Delta_{ij} \Pi_{Ui}(\{\mathbb{C}_{ij}, \widehat{\mathbb{C}}_{-ij}\})}
\Bigg]
\]
\end{footnotesize}
        \item Key terms:
        \begin{itemize}
            \item \(\Delta D_j(C)\): Gains from trade for downstream firm \(j\).
            \item \(\Delta U_i(C)\): Gains from trade for upstream firm \(i\).
            \item \(b_{ij}\): Bargaining parameter for downstream firm \(j\).
        \end{itemize}
        \item Assumes contracts of other pairs are held fixed during negotiations.
    \end{itemize}
\end{frame}
% Slide 1: Health Insurer-Hospital Negotiations - Setup
\begin{frame}{Health Insurer-Hospital Negotiations (Ho and Lee, 2017)}
    \begin{itemize}
        \item Health insurers negotiate contracts \(\mathbb{C}\) with hospitals, specifying per-admission payments \(w\), while simultaneously negotiating premiums \(\boldsymbol{p}\) with employers.
        \item Payoffs are based on Example 11 (\(\Pi_{U_i}\) for hospitals and \(\Pi_{D_j}\) for insurers).
    \end{itemize}
\end{frame}


% Slide 2: Commercial Health Insurers and Hospitals - Payoffs
\begin{frame}{Example: Health Insurers and Hospitals (Ho and Lee, 2017)}
    \begin{itemize}
        \item Hospital \(i\)'s payoff:
        \[
        \Pi_{U_i}(\mathbb{C}, \boldsymbol{p}) = \sum_{j : \mathbb{C}_{ij} \neq \mathbb{C}_0} \left(w_{ij} - c_i\right) \bar{D}_{ij}(\boldsymbol{p}, \boldsymbol{N})
        \]
        \begin{itemize}
            \item \(w_{ij}\): Payment per admission.
            \item \(c_i\): Per-admission cost.
        \end{itemize}
        
        \item Insurer \(j\)'s payoff:
        \[
        \Pi_{D_j}(\mathbb{C}, \boldsymbol{p}) = \left(p_j - \eta_j\right) \underline{D}_j(\boldsymbol{p}, \boldsymbol{N}) - \sum_{i : \mathbb{C}_{ij} \neq \mathbb{C}_0} w_{ij} \bar{D}_{ij}(\boldsymbol{p}, \boldsymbol{N})
        \]
        \begin{itemize}
            \item \(p_j\): Insurer premium.
            \item \(\eta_j\): Non-hospital costs (e.g., physician or drug payments).
            \item Two demands are different: \(\underline{D}_j(\boldsymbol{p}, \boldsymbol{N})\) and \(\bar{D}_{ij}(\boldsymbol{p}, \boldsymbol{N})\).
        \end{itemize}
        
    \end{itemize}
\end{frame}


% Slide 1: Health Insurer-Hospital Negotiations - Setup
\begin{frame}{Health Insurer-Hospital Negotiations (Ho and Lee, 2017)}
    \begin{itemize}
        \item Nash-in-Nash bargaining conditions govern hospital payments \(w_{ij}\):
        \[
            \widehat{\mathbb{C}}_{ij} \in \arg \max_{\mathbb{C}_{ij} \in \mathcal{C}_{ij}^+(\widehat{\mathbb{C}}_{-ij})}  
            \Bigg[
            {\Delta_{ij} \Pi_{Dj}(\{\mathbb{C}_{ij}, \widehat{\mathbb{C}}_{-ij}\})}^{b_{ij}}
            \cdot
            {\Delta_{ij} \Pi_{Ui}(\{\mathbb{C}_{ij}, \widehat{\mathbb{C}}_{-ij}\})}^{1-b_{ij}}
            \Bigg]
            \]
        \item Terms in the equation:
        \begin{itemize}
            \item \(\bar{D}_{ij}(\cdot)\): Number of insurer \(j\)'s enrollees admitted to hospital \(i\).
            \item \(b_{ij}\): Bargaining weight of hospital \(i\) in negotiations with insurer \(j\).
            \item \(\Delta_{ij} \Pi_{D_j}\): Insurer \(j\)'s gain from trade if the payment to hospital \(i\) is set to zero.
            \item \(\Delta_{ij} \Pi_{U_i}\): Hospital \(i\)'s gain from trade under the same condition.
        \end{itemize}
    \end{itemize}
\end{frame}

 
    % Slide 14: Modeling Contract Formation
    \begin{frame}{Modeling Contract Formation}
        \textbf{Choosing a Model:}
        \begin{itemize}
            \item The choice between TIOLI and Nash-in-Nash depends on:
            \begin{itemize}
                \item The \textbf{industry structure}: Is one party dominant, or do both have leverage?
                \item The \textbf{availability of data}: Nash-in-Nash requires more detailed data to estimate bargaining power and disagreement payoffs.
                \item The \textbf{research objective}: TIOLI is simpler and easier for theoretical models, while Nash-in-Nash is better for realistic and flexible modeling.
            \end{itemize}
        \end{itemize}
        \end{frame}
    
    

% Slide 1: Introduction to Double Marginalization
\begin{frame}{What is Double Marginalization?}
    \begin{itemize}
        \item Occurs in supply chains where a \textbf{manufacturer} and \textbf{retailer} independently maximize profits.
        \item Each adds a markup:
        \begin{itemize}
            \item Manufacturer marks up the wholesale price (\(w\)).
            \item Retailer marks up the retail price (\(p\)).
        \end{itemize}
        \item Results in:
        \begin{itemize}
            \item Higher retail price (\(p\)) than optimal.
            \item Reduced consumer demand.
            \item Lower joint profits for the supply chain.
        \end{itemize}
    \end{itemize}
\end{frame}

% Slide 2: Setup of the Problem
\begin{frame}{Setup of the Problem}
    \begin{itemize}
        \item A \textbf{manufacturer} and \textbf{retailer} negotiate the wholesale price (\(w\)) using \textbf{Nash bargaining}.
        \item The retailer simultaneously sets the retail price (\(p\)).
        \item Key assumptions:
        \begin{itemize}
            \item Nash bargaining and retail pricing are \textbf{independent but simultaneous}.
            \item The outcome depends on:
            \begin{itemize}
                \item Bargaining power (\(b\)) of the retailer.
                \item Marginal costs of the manufacturer (\(c_M\)) and retailer (\(c_R\)).
                \item Consumer demand (\(D(p)\)).
            \end{itemize}
        \end{itemize}
    \end{itemize}
\end{frame}

% Slide 3: Bargaining Equation
\begin{frame}{Wholesale Price via Nash Bargaining}
    \textbf{Nash bargaining condition:}
    \[
    \hat{w} = (1-b)(\hat{p} - c_R) + b c_M
    \]
    \begin{itemize}
        \item \(b\): Retailer's bargaining power (\(0 \leq b \leq 1\)).
        \item \(\hat{p}\): Retail price (set by the retailer).
        \item \(c_R\): Retailer's cost of selling the product.
        \item \(c_M\): Manufacturer's marginal cost of production.
    \end{itemize}
    \vspace{0.5cm}
    \textbf{Key insights:}
    \begin{itemize}
        \item If \(b = 1\): \(\hat{w} = c_M\) (retailer pays only the marginal cost).
        \item If \(b < 1\): \(\hat{w} > c_M\), leading to inefficiency (double marginalization).
    \end{itemize}
\end{frame}

% Slide 4: Retailer's Pricing Equation
\begin{frame}{Retail Price Setting}
    \textbf{Key insight:}
    \begin{itemize}
        \item Vertifical externality: when $\hat{w} > c_R$, the retailer's ignores the impact of its pricing on the manufacturer's profits.
        \item If R makes TIOLI offer, $\hat{w} = c_M$ and $\hat{p} = c_M + c_R$.
    \end{itemize}
\end{frame}

% Slide 5: Key Results and Insights
% \begin{frame}{Key Results and Insights}
%     \begin{itemize}
%         \item Double Marginalization Occurs:
%         \begin{itemize}
%             \item If \(b < 1\), the wholesale price \(\hat{w}\) exceeds the manufacturer's marginal cost (\(c_M\)).
%             \item Both the manufacturer and retailer add markups, inflating the retail price \(\hat{p}\).
%         \end{itemize}
%         \item Joint Profit is Not Maximized:
%         \begin{itemize}
%             \item High retail price reduces consumer demand.
%             \item Supply chain efficiency is reduced.
%         \end{itemize}
%         \item Effect of Bargaining Power (\(b\)):
%         \begin{itemize}
%             \item Higher \(b \Rightarrow \hat{w} \approx c_M\), reducing inefficiency.
%             \item Lower $b \Rightarrow$ Greater inefficiency.
%         \end{itemize}
%     \end{itemize}
% \end{frame}


% Slide 19: Estimation and Identification
\begin{frame}{Supply Estimation and Identification}
    \begin{itemize}
        \item Estimation involves recovering key parameters:
        \begin{itemize}
            \item Marginal costs (\(c^U, c^D\)).
            \item Bargaining parameters (\(b_{ij}\)).
            \item Gains from trade (\(\Delta D_j, \Delta U_i\)).
        \end{itemize}
        \item Ideal data scenario:
        \begin{itemize}
            \item Observed wholesale prices (\(w\)).
            \item Observed demand system (\(D(p)\)).
            \item Marginal cost $c_R$ and $c_M$.
        \end{itemize}
        \item Missing data (e.g., marginal costs) requires additional assumptions or instruments.
    \end{itemize}
\end{frame}

% Slide 21: Demand Estimation
\begin{frame}{Demand Estimation}
    \begin{itemize}
        \item A key input for vertical contracting models is estimating consumer demand.
        \item Demand estimation helps predict:
        \begin{itemize}
            \item Upstream and downstream quantities (\(D(p, x, w)\)).
            \item Consumer responses to prices and product characteristics (\(x\)).
        \end{itemize}
        \item Techniques:
        \begin{itemize}
            \item Use exogenous variation in prices and characteristics to identify demand.
            \item Estimate demand functions for both upstream and downstream firms.
        \end{itemize}
    \end{itemize}
\end{frame}

% Slide 22: Example - Estimating Demand in Healthcare
\begin{frame}{Example: Estimating Demand in Healthcare}
    \begin{itemize}
        \item Ho and Lee (2017): Model demand for health insurance plans and hospital services.
        \item Consumer utility for insurer \(j\):
        \[
        u_{cjm} = \beta v_{cjm} + x_{jm} \beta_x + \xi_{jm} + \epsilon_{cjm}
        \]
        \begin{itemize}
            \item \(v_{cjm}\): Willingness to pay (WTP) for insurer \(j\)'s hospital network.
            \item \(x_{jm}\): Observed characteristics (e.g., premiums).
            \item \(\xi_{jm}\): Unobserved demand shocks.
            \item \(\epsilon_{cjm}\): Idiosyncratic preferences.
        \end{itemize}
        \item Model jointly estimates insurer and hospital demand.
    \end{itemize}
\end{frame}

% Slide 23: Usage Models for Bundles
\begin{frame}{Usage Models for Bundles}
    \begin{itemize}
        \item When consumers purchase bundles, usage data can inform valuation.
        \item Example: Multichannel TV (Crawford et al., 2018)
        \begin{itemize}
            \item Consumer utility for distributor \(j\):
            \[
            u_{cjm} = \beta v_{cjm}(C_j) + x_{jm} \beta_x + \xi_{jm} + \epsilon_{cjm}
            \]
            \item \(v_{cjm}(C_j)\): Viewership utility for channels in bundle \(C_j\).
            \item Viewership data helps estimate valuations for individual channels.
        \end{itemize}
        \item Usage models reduce data requirements by linking upstream and downstream choices.
    \end{itemize}
\end{frame}

% Slide 24: Upstream Choice-Only Models
\begin{frame}{Upstream Choice-Only Models}
    \begin{itemize}
        \item In some cases, only upstream demand is modeled.
        \item Example: Grennan (2013) - Medical devices
        \begin{itemize}
            \item Focuses on hospitals' choice of medical devices.
            \item Does not model patient flows across hospitals.
        \end{itemize}
        \item Simplifies computation but ignores some competitive effects.
    \end{itemize}
\end{frame}

% Slide 25: Consumer Selection
\begin{frame}{Consumer Selection in Demand Estimation}
    \begin{itemize}
        \item Selection bias arises when observed consumption depends on unobserved preferences.
        \item Example: Multichannel TV (Crawford and Yurukoglu, 2012)
        \begin{itemize}
            \item Consumers who purchase bundles may have higher valuations for included channels.
            \item Ignoring selection leads to overestimating valuations.
        \end{itemize}
        \item Solution:
        \begin{itemize}
            \item Jointly estimate upstream and downstream demand.
        \end{itemize}
    \end{itemize}
\end{frame}

% Slide 26: Joint Estimation of Demand and Supply
\begin{frame}{Joint Estimation of Demand and Supply}
    \begin{itemize}
        \item Demand and supply parameters can be jointly estimated for efficiency.
        \item Example: Crawford et al. (2018)
        \begin{itemize}
            \item Wholesale prices (\(w_{ij}\)) used to infer demand valuations.
            \item Assumes content with higher fees has higher consumer value.
        \end{itemize}
        \item Benefits:
        \begin{itemize}
            \item Increases precision of demand estimates.
            \item Captures interactions between demand and supply.
        \end{itemize}
    \end{itemize}
\end{frame}

% Slide 27: Conclusion
\begin{frame}{Conclusion}
    \begin{itemize}
        \item Vertical contracting models capture interactions between upstream and downstream firms.
        \item Estimation requires careful modeling of demand and supply.
        \item Techniques like Nash bargaining and demand estimation help recover key parameters.
        \item Joint estimation of demand and supply can improve precision and capture interactions.
    \end{itemize}

    \vspace{0.5cm}
    \begin{center}
        \textbf{Thank you!}
    \end{center}
\end{frame}
\end{document} 